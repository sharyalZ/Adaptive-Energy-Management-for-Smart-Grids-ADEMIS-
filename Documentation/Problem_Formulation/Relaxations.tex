The original power flow is a nonconvex QCP problem. Linearization greatly simplifies the formulation, but optimality of the solution is compromised. In comparison, relaxations of the constraints provide a better solution. In this section three such relaxations are highlighted namely, semidefinite programming, convex semidefinite programming and second order cone programming. The relaxations are listed in increasing order of efficiency and tractability. 

\subsubsection{Semidefinite Programming:}

As metioned earlier, the main source of nonconvexity is equation (2). A \textit{lift and project} methodology is used to relax this constraint. To do so, a Hermitian matrix $V$ is assumed where $V=vv^\ast$. Equivalently this constraint can be written in semidefinite form as $V\succeq0$ and $rank(V)=1$. Hence , the new relaxed feasible set is listed below:

\begin{tcolorbox}[colback=gray!5!white,colframe=gray!25!black,title=Feasible Set 4 (Nonconvex Semidefinite Power Flow):]
\[
Objective Function\hspace{0.1cm}(1)
\]

\[
  p_{ab}(t) + jq_{ab}(t)= \left ( V_{aa} - V_{ab} \right )y_{ab}^\ast
\]

\[
  v_{a,min}^2\leq \left |V_{aa}(t) \right | \leq v_{a,max}^2  
\]

\[
    V\succeq0
\]

\[
    rank(V)=1
\]


\[
Eq.(3) - Eq.(7)\hspace{0.1cm} \&\hspace{0.1cm} Eq.(9) - Eq.(17) \hspace{0.5cm} [Power \hspace{0.1cm} Flow \hspace{0.1cm}  Constraints] 
\]

\end{tcolorbox}

This simplifies the equation (2) from a solver's point of view, but the nonconvexity is still a part of this system. The rank constraint of a matrix is also a nonconvex constraint(as $rank(\lambda A + (1-\lambda B) \neq 1$, where  $rank(A) =1$, $rank(B) = 1$ and $\lambda=(0,1)$).
\subsubsection{Convex Semidefinite Programming:}

The nonconvexity of the rank constraint can be solved by simply ignoring the rank constraint. Hence, if the solution is obtained with the formulation of feasible set 5, and if the solution happens to satisfy the rank constraint then it is also an optimal solution for feasible set 4. As, feasible set 4 is equivalent to original nonconvex problem so the solution optimal for it would also be optimal for the original nonconvex problem.  

\begin{tcolorbox}[colback=gray!5!white,colframe=gray!25!black,title=Feasible Set 5 (Convex Semidefinite Power Flow):]
\[
Objective Function\hspace{0.1cm}(1)
\]

\[
  p_{ab}(t) + jq_{ab}(t)= \left ( V_{aa} - V_{ab} \right )y_{ab}^\ast
\]

\[
  v_{a,min}^2\leq \left |V_{aa}(t) \right | \leq v_{a,max}^2  
\]

\[
    V\succeq0
\]

\[
Eq.(3) - Eq.(7)\hspace{0.1cm} \&\hspace{0.1cm} Eq.(9) - Eq.(17) \hspace{0.5cm} [Power \hspace{0.1cm} Flow \hspace{0.1cm}  Constraints] 
\]
\end{tcolorbox}

The feasible set 5 is nonconvex and can be solved easily using commercial solvers. If voltage values are desired then the matrix $V$ can be decomposed into lower triangular matrix and its conjugate using Cholesky decomposition i.e. $V=vv^\ast$. If the angles of the voltages are also desired then concept of principal eigenvectors can be exploited. Generally, the largest eigenvalue of the Hermitian matrix $V$ would provide a good approximation of the angles. Principal eigenvector can be obtained through: 

\[
\lambda x = Vx
\]

\[
\lambda x = vv^\ast x
\]

\subsubsection{Second Order Cone Programming:}

A more efficient and tractable relaxation is the second order cone programming(SOCP). The positive semidefinite constraint can be relaxed to a second order cone approximation. The new SOCP feasible set is described below:

\begin{tcolorbox}[colback=gray!5!white,colframe=gray!25!black,title=Feasible Set 6 (Second Order Cone Power Flow):]
\[
Objective Function\hspace{0.1cm}(1)
\]

\[
  p_{ab}(t) + jq_{ab}(t)= \left ( V_{aa} - V_{ab} \right )y_{ab}^\ast
\]

\[
  v_{a,min}^2\leq \left |V_{aa}(t) \right | \leq v_{a,max}^2  
\]

\[
 \left.\begin{aligned}
    V_{ab}V_{ab}^\ast \leq V_{aa}V_{bb} \\
    V_{aa}\geq 0 \\
    V_{bb}\geq 0
       \end{aligned}
 \right\}
 \qquad \text{Hyperbolic Constraints}
\]

\[
Eq.(3) - Eq.(7)\hspace{0.1cm} \&\hspace{0.1cm} Eq.(9) - Eq.(17) \hspace{0.5cm} [Power \hspace{0.1cm} Flow \hspace{0.1cm}  Constraints] 
\]

\end{tcolorbox}

The hyperbolic constraint here can be written in the standard SOC form i.e. $\lVert Ax+b \rVert \leq c^Tx+d$, as: 

\[
   \left \|  \begin{bmatrix}
        2\sqrt{V_{ab}V_{ba}^\ast} \\
        V_{aa}-V_{bb} 
            \end{bmatrix} 
    \right \| 
    \leq V_{aa}+V_{bb}
\]